\documentclass{article}

\usepackage[margin=1in]{geometry}
\usepackage{amsfonts}
\usepackage{amsmath}
\usepackage{bchart}
\usepackage{caption}
\usepackage{graphicx}
\usepackage{newclude}
\usepackage{parskip}
\usepackage{pgfplots}
\usepackage{subfig}
\usepackage{titling}

% \usepackage{tikz}
% \usetikzlibrary{matrix}
% \usetikzlibrary{positioning}
%\usetikzlibrary{arrows}

\begin{document}

\title{AI Project 2: Genetic Algorithm}
\author{Roy Howie}
\date{February 27, 2017}
\maketitle

% \begin{abstract}
%   Summary.
% \end{abstract}

\section{The Genetic Algorithm}
  The Genetic Algorithm is a stochastic, heuristic-based algorithm which draws
  its inspiration from the Darwinian principle of natural selection. It aims to
  generate high-quality but not necessarily maximal solutions to optimization
  and search problems.

  The algorithm requires an initial population and a fitness function $f$ which
  it seeks to maximize. The initial population is usually a collection of
  individuals which have each been given a random genetic makeup. The algorithm
  is then able to efficiently search through extremely large domains via three
  search operators:
  \begin{itemize}
    \renewcommand\labelitemi{}
    \item{
      \textbf{Survival.} This determines whether a given individual lives on to
      the next generation.
    }
    \item{
      \textbf{Recombination.} Individuals selected for survival are mated and
      their genetic compositions mixed.
    }
    \item{
      \textbf{Mutation.} A chromosome is occasionally incorrectly copied when
      passed onto the next generation.
    }
  \end{itemize}
  The application of these three search operators to a population is the act of
  moving from one generation to the next.

  Each of these three operators can be customized for whichever domain is under
  consideration.

  For example, the \textbf{survival} operator could be changed to always include
  the top quintile in its selection. The \textbf{recombination} operator could
  randomly select the offsprings' genes from each parent or it could ensure an
  even split. Instead of remaining constant, the \textbf{mutation} rate can be
  intensified or lessened over time. The possibilities are endless.

  Last, the algorithm requires a termination criterion, such as having
  attained a maximal level of fitness within a population or having run for a
  given number of generations. This termination criterion must take into account
  the possibility of \textbf{premature degeneration}. This is when a population
  devolves into a small set of individuals which are no longer capable of
  genetic diversity through mating, i.e. recombination. Alternatively, the
  mutation rate may be momentarily increased to temporarily overcome this
  affliction.

\end{document}
