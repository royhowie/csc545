\documentclass{article}

\usepackage[margin=1in]{geometry}
\usepackage{amsfonts}
\usepackage{amsmath}
\usepackage{parskip}
\usepackage{newclude}
% \usepackage{enumitem}
\usepackage{graphicx}
\usepackage{caption}

\usepackage{bchart}
\usepackage{pgfplots}

\usepackage{tikz}
\usetikzlibrary{matrix}
\usetikzlibrary{positioning}
%\usetikzlibrary{arrows}

\begin{document}

% \title{Sliding Tiles\\\large AI Project 1}
\title{AI Project 1: Sliding Tiles}
\author{Roy Howie}
\date{February 8, 2017}
\maketitle

\section*{The Puzzle}
  \subsection*{History}
    The 15-puzzle harkens back to the late 19th century, when it was invented by
    Noyes Chapman, the Postmaster of Canastota, New York, who applied for a
    patent in March of 1880.

    It is often misattributed, however, to Sam Loyd (1841-1911), who is widely
    known as being one of America's greatest puzzle-writers. This is due to
    Loyd's (debunked) misinformation campaign---beginning in 1891 and continuing
    until his death 20 years later---that he was the puzzle's sole inventor.

    The puzzle, along with the smaller 8-puzzle variant, has been enjoyed
    throughout the years for its simplicity. The 15-puzzle is sometimes called
    the 16-puzzle; similarly, the 8-puzzle is occassionally referred to as the
    9-puzzle. It is but a matter of semantics: to count the blank tile or not?

  \subsection*{How it works}
    The concept is simple. Given a starting state---such as the one below---move
    tiles horizontally or vertically into the blank space until the goal state
    is reached. For example, in figure \ref{fig:start1}, the player would first
    move the \textbf{14} tile to the left. Next, the player would move the
    \textbf{15} tile to the left. This last action would complete the puzzle.

    But how does a player know whether a puzzle is solvable? Indeed, figure
    \ref{fig:bad15} presents a puzzle state which is unsolvable, i.e., one which
    can never be permuted into the ``goal state'' represented by figure
    \ref{fig:goal15}.

  \include*{fig1-3}

  \subsection*{Determining a puzzle's solvability}
    As there are $16!=20,922,789,888,000$ different puzzle board arrangements,
    listing which starting arrangements are good and which are bad is
    clearly unfeasible. Indeed, half of all possible arrangements---about
    10 trillion---cannot be permuted into the goal state represented in figure
    \ref{fig:goal15}. For convenience, we shall henceforth take ``solvable'' to
    mean ``can be permuted into the (standard) goal state represented in figure
    \ref{fig:goal15}''.

    A puzzle state is unsolvable if it of \textbf{odd} permutation, as proven by
    W.W. Johnson in 1879. Conversely, a puzzle state is solvable if it is of
    \textbf{even} permutation, as proven by W.E. Story in 1879.

    To determine the permutation of a puzzle, consider the sum of its tile
    ``inversions.'' For example, let $P$ be a puzzle. Consider a tile $T_i\in P$
    containing the number $i$. Reading from left to right and from top to
    bottom, let $n_i$ be the number of tiles with numbers less than $i$ which
    occur after $T_i$. The blank tile is ignored. Let $N$ be the sum of
    inversions:

    $$
      N=\sum_{i=2}^{15} n_i
    $$

    Note that the sum begins at $i=2$, as no tiles contain numbers of value
    less than $1$.

    Next, let $e$ be the row number of the blank tile. In $P$, the blank tile
    is in the bottom-right corner, or the 4th row, so $e=4$. Hence,
    $\text{Permutation}(P)=N+e\pmod{2}$. Thus, if $N+e$ is even, the puzzle is
    deemed solvable; otherwise, the puzzle is unsolvable.

    \include*{fig4}

    For example, consider the puzzle state represented in figure
    \ref{fig:example-state}. We then have that $n_6=3$, as three tiles of value
    less than six occur after $T_6$: $T_2$, $T_3$, and $T_4$. Here are the $n_i$
    for all $T_i$ in figure \ref{fig:example-state}.

    \include*{table1}

    Thus, $\text{Permutation}(P)= N + e= 50$, which is even, so the puzzle
    represented in figure \ref{fig:example-state} is solvable.

  \subsection*{Generalizing puzzle solvability to other dimensions}
    As mentioned previously, other similar forms of the 15-puzzle exist, such as
    the smaller 8-puzzle. In general, if a puzzle is of dimension $k\times k$,
    determining its solvability is not much complicated than the process
    enumerated above for the 15-puzzle of size $4\times 4$.

    Let $N$ be the sum of tile inversions $n_i$, as before, and let $e$ be the
    row number of the blank tile. Consider a puzzle $P$ of size $k\times k$.
    When $k$ is odd, $P$ is solvable if $N$ is even; when $k$ is even, $P$ is
    solvable when $N+e$ is even.

    Indeed, one can even consider non-square puzzles of size $k\times l$, but
    that is outside the scope of this paper.

\section*{Solving the puzzle}
  \subsection*{Goal}
    Given a \textit{valid} puzzle starting state, find a path of individual tile
    movements which leads to the goal state. A \textit{valid} puzzle is said to
    be one which has an even permutation and can, therefore, be permuted into
    the standard goal state.

  \subsection*{Constraints}
    Due to the prohibitive amount of states---over 20 trillion---the 15-puzzle
    can inhabit and the corresponding amount of memory and time a personal
    computer would require to process said states, we will consider the smaller
    8-puzzle, as it has only $9!=362,880$ different states.

  \subsection*{Approach}
    We will create a search tree out of the puzzle. The search operator will
    be all previously unseen moves available to a given state via sliding a
    single tile. The root node will be a random starting state.

    The puzzle states will be (conveniently) stored in memory as hashes; this
    will allow for easy lookup in a set to determine whether a state has already
    been visited and should therefore be passed over by the search operator.
    For example, the puzzle state from figure \ref{fig:example-state} would
    be represented by the hash \texttt{187d56af2ec0394b}. This is obtained
    by reading from left to right and from top to bottom, considering the
    numbers in base-16; the blank tile is read as zero.

  \subsection*{Algorithms}
    The search tree will be searched using DFS (depth-first search), BFS
    (breadth-first search), and A*. Iterative deepening search was determined
    not to be suitable to the puzzle, as it resulted in a search tree no
    different from BFS; this is due to the nature of the search tree, which will
    be explained in detail later.

  \subsection*{Rubric}
    The three algorithms will be run for 1300 trials each. Each will be
    evaluated based on the the length of the closed list (total number of nodes
    visited), the maximum length of the open list (maximum number of nodes
    present at any point in the search tree), and the depth of the tree to the
    desired goal state (the number of tiles moved by the algorithm to solve the
    puzzle).

\pagebreak
\section*{Results}
  \include*{graphs}

\section*{Discussion}
  DFS was by far the least adequate search algorithm tasked with finding a
  solution to the 8-puzzle. A* and BFS were much closer in their performance,
  with BFS providing a much more practical solution. A* used the Taxicab Metric
  as its heuristic; this heuristic is known to be admissable, but perhaps there
  are better, more sophisticated heuristics available. Thus, if the algorithms
  were ranked based on the abstract concept of ``usability,'' they would be
  placed in the following order: BFS, A*, DFS.

  The search tree for the sliding tiles puzzles has a low branching factor of
  zero to three---the fourth option is always corresponds to the last
  state---but
  an extremely high depth. This was most likely the reason for the poor showing
  by DFS. As DFS traverses the tree, plunging downwards, it skips very few
  states. In fact,
  it is very likely to hit the children of previously unexplored children, as
  the graph of puzzle states is highly connected, i.e., a puzzle state can be
  easily reached from most other states with very few intermediary steps.

  This resulted in DFS's ambling about, exploring long, winding ``caverns'' or
  ``tunnels'' until it either reached a cul-de-sac or the solution. Upon
  reaching a dead-end, DFS would simply backtrack, beginning its pursuit anew.
  This ultimately caused DFS to have an average solution depth of over 60
  thousand; compare this to the average solution depths of A* and BFS at
  250 and 23, respectively. Hence, DFS is completely impractical, as no human
  would suffer through so many moves to solve the simple 8-puzzle. Indeed, even
  A* would be unbearable with its 250 moves, as most players would easily find a
  solution requiring a number of steps on an order of magnitude similar to that
  of BFS.

  Ignoring the practicality of the solutions presented by the three search
  algorithms, the results were much more even-handed, but still lopsided.
  A* and DFS both had an average open list size of around 41k, whereas BFS had
  an average open list size of 25k. This was a surprisingly result, as BFS
  typically requires more memory to store the search tree, i.e., has a larger
  open tree size.

  As previously seen states were not revisited by the
  algorithms, BFS most likely had a smaller search tree because it more quickly
  eliminated dead-ends and bad choices, preventing those paths from being
  searched. This was also the reason the Iterativing Deepening Search algorithm
  was not included, as it was practically the same as BFS due to the low
  branching factor and the small depth of potential solution tree (most
  8-puzzles can be solved in under 30 moves).

  On the other hand, A* and DFS visited about 70\% as many states as BFS did,
  with 65k and 92k states visited on average, respectively. Indeed, BFS took,
  on average, about 38\% longer to complete its search than A* and DFS did.

  Despite being somewhat slower, BFS was the clear winner, as it was able to
  present a human-understandable solution using less memory.

\vfill
\section*{Bibliography}
  \begin{itemize}
    \item http://mathworld.wolfram.com/15Puzzle.html
    \item http://www.geeksforgeeks.org/check-instance-15-puzzle-solvable/
    \item http://www.geeksforgeeks.org/check-instance-8-puzzle-solvable/
  \end{itemize}

\end{document}
